% !TEX TS-program = pdflatex
% !TEX encoding = UTF-8 Unicode
\documentclass[11pt]{amsart} 

\usepackage[utf8]{inputenc}
\usepackage{geometry}
\geometry{letterpaper}
\usepackage{graphicx}
\usepackage[parfill]{
	parskip
}

%%% PACKAGES
\usepackage{amsmath}
\usepackage{amssymb}
\usepackage{sistyle}
\SIthousandsep{,}
\usepackage{hyperref}
\usepackage{biblatex}
\usepackage{xurl}

%%% Theorem-like environments
\newtheorem{theorem}{Theorem}
\newtheorem{corollary}{Corollary}

%%% Macros
\DeclareMathOperator*{\lcm}{\mathrm{lcm}}

\addbibresource{growth-of-digital-sums-of-powers.bib} %Import the bibliography file



\title{Elementary Bounds on Digital Sums of Powers, Factorials, and LCMs}
\author{David G Radcliffe}
\date{\today}

\begin{document}

	\begin{abstract}
		We prove that the sum of the base-$b$ digits of $a^{n}$ grows at least logarithmically
		in $n$, provided that the exponents in the prime factorization of $b$ are
		not uniformly proportional to the corresponding exponents in the prime
		factorization of $a$. Our approach uses only elementary number theory and applies
		to a wide class of sequences, including factorials and
		$\Lambda(n) = \lcm(1, 2, \ldots, n)$.
	\end{abstract}

	\maketitle

	\section{Motivation}

	This article was inspired by the following problem, which was posed and solved
	by Wacław Sierpiński\cite[Problem 209]{sierpinski1970}:
	\begin{quote}
		\emph{Prove that the sum of digits of the number $2^{n}$ (in decimal system)
		increases to infinity with $n$.}
	\end{quote}

	We will prove a sufficient condition on positive integers $a$ and $b$ which implies
	that the sum of the base-$b$ digits of $a^{n}$ grows at least logarithmically in
	$n$. This condition includes Sierpiński's problem as a special case. The asymptotic 
  behavior of digital sums of powers has been studied by Senge and Straus (1973) and Stewart
	(1980), but their results rely on methods from transcendence theory. In contrast,
	the arguments here are elementary.

	Consider the sequence of powers of 2 (sequence \href{https://oeis.org/A000079}{A000079} 
  in the OEIS):

	\[
		1, 2, 4, 8, 16, 32, 64, 128, 256, 512, 1024, \ldots .
	\]

	This sequence grows very rapidly. Now, let us define another sequence by adding
	the decimal digits of each power of 2. For example, 16 becomes $1+6=7$, and 32
	becomes $3+2=5$. The first few terms of this new sequence (\href{https://oeis.org/A001370}{A001370}) are listed below.

	\[
		1, 2, 4, 8, 7, 5, 10, 11, 13, 8, 7, \ldots .
	\]

	It is apparent that this sequence grows much more slowly, and it is not
	monotone. Nevertheless, it is reasonable to conjecture that the sequence
	diverges to infinity. Indeed, we should expect that the sum of the decimal
	digits of $2^{n}$ is asymptotic to $4.5 n \log_{10}2$, since $2^{n}$ has $\lfloor
	n \log_{10}2\rfloor + 1$ decimal digits and the digits seem to be approximately
	uniformly distributed among $0, 1, 2, \ldots, 9$. However, this stronger conjecture
	remains to be proved.

	For an integer $b \ge 2$, we write $s_{b}(n)$ for the sum of the base-$b$
	digits of $n$, and $c_{b}(n)$ for the number of nonzero digits in that expansion.
	These functions are asymptotic to each other, since $c_{b}(n) \le s_{b}(n) \le
	(b-1) c_{b}(n)$ for all $n$ and $b$, so we will restrict our attention to
	$c_{b}(n)$. For a prime $p$, $\nu_{p}(n)$ denotes the exponent of $p$ in the
	prime factorization of $n$.

	Let us prove that $\lim\limits_{n\to\infty}c_{10}(2^{n}) = \infty$. Observe that
	the final digit of $2^{n}$ cannot be 0, since only multiples of 10 can end in
	0.

	If $2^{n}$ has four or more digits, then the last four digits cannot start
	with three consecutive zeros. This is because $2^{n}$ is divisible by 16, so
	$2^{n} \bmod 10^{4}$, the number formed by the last four digits of $2^{n}$, is
	also divisible by 16. If the first three of these digits were zero, then $2^{n}
	\bmod 10^{4}$ would be less than 10, which is a contradiction.

	Similarly, if $2^{n}$ has 14 or more digits, then
	$2^{n} \bmod 10^{14}\ge 2^{14}> 10^{4}$. So the last 14 digits of $2^{n}$ cannot
	start with 10 consecutive zeros.

	We can continue in this way, finding longer and longer non-overlapping blocks
	of digits, each containing at least one nonzero digit. This shows that as $n$
	increases to infinity, the number of nonzero digits of $2^{n}$ also increases to
	infinity. Let us formalize this argument.

	\begin{theorem}
		Let $\{\alpha_{k}\}$ be a sequence of positive integers such that $\alpha_{1} = 1$ and
		$2^{\alpha_{k+1}}> 10^{\alpha_k}$ for all $k \ge 1$. If $n$ is a positive integer that
		is divisible by $2^{\alpha_k}$ but not divisible by 10, then $c_{10}(n) \ge k$.
		\label{base-ten-lower-bound}
	\end{theorem}

	\begin{proof}
		The proof is by induction on $k$. The case $k = 1$ is trivial, so let us assume
		that $k \ge 2$. By the division algorithm, there exist integers $q \ge 0$ and
		$0 \le r < 10^{\alpha_{k-1}}$ such that

		\[
			n = 10^{\alpha_{k-1}}q + r.
		\]

		Since $n \ge 2^{\alpha_k}> 10^{\alpha_{k-1}}$, it follows that $q \ge 1$.

		Since $n$ and $10^{\alpha_{k-1}}$ are divisible by $2^{\alpha_{k-1}}$, $r$ is also divisible
		by $2^{\alpha_{k-1}}$. Moreover, $r$ is not divisible by 10, so
		$c_{10}(r) \ge k - 1$ by the induction hypothesis.

		Note that $c_{10}(n) = c_{10}(q) + c_{10}(r)$, since the digit expansion of $n$
		is the concatenation of the digit expansions of $q$ and $r$, possibly with
		leading zeros. Therefore,
		\[
			c_{10}(n) = c_{10}(q) + c_{10}(r) \ge 1 + (k - 1) = k.
		\]
	\end{proof}

	\begin{corollary}
		Let $a$ be a positive integer that is divisible by 2 but not divisible by 10.
		Then $\lim\limits_{n\to \infty}c_{10}(a^{n}) = \infty$.
		\label{even-powers-base-ten-limit}
	\end{corollary}

	\begin{proof}
		Let $k$ be a positive integer. If $n \ge \alpha_{k}$ then $a^{n}$ is divisible by
		$2^{\alpha_k}$ but not divisible by 10, so $c_{10}(a^{n}) \ge k$ by the previous theorem.
		Therefore, $c_{10}(a^{n}) \ge k$ for all $n \ge \alpha_{k}$. Since $k$ is
		arbitrary, we conclude that $\lim\limits_{n\to\infty}c_{10}(a^{n}) = \infty$.
	\end{proof}

	\section{Generalizing to other bases}

	Our proofs rely only on divisibility properties and therefore extend naturally
	to arbitrary bases.

	\begin{theorem}
		Let $b \ge 2$ be an integer that is not a power of a prime, and let $p$ be a
		prime divisor of $b$. Let $\{\alpha_{k}\}$ be a sequence of positive integers such
		that $\alpha_{1} = 1$ and $p^{\alpha_{k+1}}> b^{\alpha_k}$ for all $k \ge 1$. If $n$ is a
		positive integer that is divisible by $p^{\alpha_k}$ but not divisible by $b$, then
		$c_{b}(n) \ge k$. \label{general-powers-lower-bound}
	\end{theorem}

	\begin{proof}
		The proof is by induction on $k$. The case $k = 1$ is trivial, so let us assume
		that $k \ge 2$. By the division algorithm, there exist integers $q \ge 0$ and
		$0 \le r < b^{\alpha_{k-1}}$ such that
		\[
			n = b^{\alpha_{k-1}}q + r.
		\]

		Since $n \ge p^{\alpha_k}> b^{\alpha_{k-1}}$, it follows that $q \ge 1$.

		Since $n$ and $b^{\alpha_{k-1}}$ are divisible by $p^{\alpha_{k-1}}$, $r$ is also
		divisible by $p^{\alpha_{k-1}}$. Moreover, $r$ is not divisible by $b$, so $c_{b}(
		r) \ge k - 1$ by the induction hypothesis.

		Therefore,
		\[
			c_{b}(n) = c_{b}(q) + c_{b}(r) \ge 1 + (k - 1) = k.
		\]
	\end{proof}

	\begin{theorem}
		Let $b \ge 2$ be an integer that is not a power of a prime, let $p$ and $q$
		be distinct prime divisors of $b$, and let $\{\alpha_{k}\}$ be defined as in Theorem
		2. If
		\[
			\phi_{b,p,q}(n) := \nu_{p}(n) - \nu_{q}(n) \frac{\nu_{p}(b)}{\nu_{q}(b)}\ge \alpha_{k}
		\]
		then $c_{p}(n) \ge k$. \label{phi-lower-bound}
	\end{theorem}

	\begin{proof}
		Write $n$ as $b^{r} m$, where $m$ is not divisible by $b$. The function $\phi
		_{b,p,q}$ satisfies $\phi_{b,p,q}(b) = 0$ and $\phi_{b,p,q}(uv) = \phi_{b,p,q}(u) + \phi
		_{b,p,q}(v)$ for all positive integers $u$ and $v$, so
		$\phi_{b,p,q}(n) = \phi_{b,p,q}(m)$. Since $\nu_{p}(m) \ge \phi_{b,p,q}(m) \ge \alpha_{k}$
		and $b \nmid m$, Theorem \ref{general-powers-lower-bound} implies that
		$c_{b}(m) \ge k$. But $c_{b}(m) = c_{b}(n)$, since $m$ and $n$ have the same
		digits in base $b$, apart from trailing zeros. Therefore, $c_{b}(n) \ge k$.
	\end{proof}

	\begin{theorem}
		Let $a \ge 2$ and $b \ge 2$ be integers. Suppose that $\log(\frac{a}{d}) / \log(b)$
		is irrational, where $d$ is the largest factor of $a$ that is coprime to $b$.
		Then $\lim\limits_{n\to\infty}c_{b}(a^{n}) = \infty$.
	\end{theorem}

	\begin{proof}
    Let $p_1^{e_1} \cdots p_r^{e_r}$ be the prime factorization of $b$.
    Then $\frac{a}{d} = p_1^{f_1} \cdots p_r^{f_r}$, where some of the $f_i$ may be zero.
    If 
    \[
      \frac{f_1}{e_1} = \cdots = \frac{f_r}{e_r} = t
    \]
    then $\log(\frac{a}{d}) / \log(b) = t$, which is rational.

    Therefore, if $\log(\frac{a}{d}) / \log(b)$ is irrational, then the ratios
    $f_i/e_i$ are not all equal, which implies that $b$ has two prime
    factors $p = p_i$ and $q = p_j$ such that $f_i / e_i > f_j / e_j$, 
    hence $\phi_{b,p,q}(a) > 0$.
  
		Let $k > 0$ be given. There exists an integer $N$ such that
		$\phi_{b,p,q}(a^{n}) = n \phi_{b,p,q}(a) \ge \alpha_{k}$ for all $n \ge N$, so Theorem
		\ref{phi-lower-bound} implies that $c_{b}(a^{n}) \ge k$ for all $n \ge N$.
		Since $k$ is arbitrary, we conclude that $\lim\limits_{n\to\infty}c_{b}(a^{n}
		) = \infty$.
	\end{proof}

	Note that this argument also implies that $c_{b}(a^{n}) > C \log n$,
  where $C$ depends on $a$ and $b$ alone. 
  In 1973, Senge and Straus\cite[Theorem 3]{senge-straus1973}
	proved that if $a \ge 1$ and $b \ge 2$ are positive integers, then
	$\lim\limits_{n \to \infty}c_{b}(a^{n}) = \infty$ if and only if
	$\log(a) / \log(b)$ is irrational. However, they did not demonstrate a lower bound. 
  In 1980, Stewart\cite[Theorem 2]{stewart1980} proved that
	if $\log(a)/\log(b)$ is irrational then
	\[
		c_{b}(a^{n}) > \frac{\log n}{\log \log n + C} - 1
	\]
	for $n > 4$, where $C$ depends on $a$ and $b$ alone.

	\section{Related sequences}

	Theorem 3 can be applied to many other sequences. We will give two examples
	here. As before, let $b \ge 2$ be an integer that is not a prime power, let $p$
	and $q$ be distinct prime divisors of $b$, and let
	\[
		\phi_{b,p,q}(n) = \nu_{p}(n) - \nu_{q}(n) \frac{\nu_{p}(b)}{\nu_{q}(b)}.
	\]

	\subsection*{Factorials}

	By Legendre's formula\cite[p. 263]{dickson1919},
	$\nu_{p}(n!) = (n - s_{p}(n)) / (p - 1)$. Thus,
	\[
		\phi_{b,p, q}(n!) \approx n \left( \frac{1}{p-1}- \frac{\nu_{p}(b)}{(q-1)\nu_{q}(b)}
		\right)
	\]
	whenever $(p-1) \nu_{p}(b) \ne (q-1) \nu_{q}(b)$, since $s_{p}(n) + s_{q}(n) \ll
	n$. Therefore, $\lim\limits_{n\to\infty}c_{b}(n!) = \infty$ for any $b$ with distinct
	factors $p$ and $q$ satisfying
	\[
		(p-1) \nu_{p}(b) > (q-1) \nu_{q}(b).
	\]
	In particular, $\lim\limits_{n\to\infty}c_{10}(n!) = \infty$.

	\subsection*{Cumulative LCMs}

	Let $\Lambda_{n} = \lcm(1, 2, \ldots, n)$. It is easy to see that
	$\nu_{p}(\Lambda_{n}) = \lfloor \log_{p}(n) \rfloor$. Thus,
	\[
		\phi_{b,p,q}(\Lambda_{n}) = \lfloor \log_{p} n\rfloor - \lfloor \log_{q} n\rfloor
		\cdot \frac{\nu_{p}(b)}{\nu_{q}(b)}\approx \log n \cdot \left( \frac{1}{\log
		p}- \frac{1}{\log q}\cdot \frac{\nu_{p}(b)}{\nu_{q}(b)}\right).
	\]

	The quantity in parentheses is nonzero, since $\log(p)/\log(q)$ is irrational,
	and we may assume that it is positive, else we can switch $p$ and $q$. Therefore,
	$\lim\limits_{n\to\infty}\phi_{b,p,q}(\Lambda_{n}) = \infty$, hence $\lim\limits_{n
	\to \infty}c_{b}(\Lambda_{n}) = \infty$ for every base $b$ that is not a prime
	power.

	{\bf Remark.} Sanna\cite{sanna2015} proved that $s_{b}(n!)$ and
	$s_{b}(\Lambda_{n})$ are greater than $C \log n \log \log \log n$ for every
	integer $n > e^{e}$ and every $b \ge 2$, where $C$ is a constant depending only
	on $b$.

	\medskip

	\printbibliography
\end{document}